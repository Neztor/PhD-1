\section{Summary, Conclusions, and Outlook}

To facilitate the decision of whether a magnetized wall is necessary in addition to the five muon spoilers to guarantee acceptable muon shielding, the muons from the MUCARLO simulation have been simulated in a full SiD detector simulation.
In particular, the differences between the two shielding scenarios, and in both cases the impact of the muon background on the SiD occupancy have been studied.

The simulation showed that low energy muons from the Beam Delivery System are stopped or deflected by the magnetized wall, because of which the muon rate is reduced by a factor of about 6.
Additionally, the spatial distributions of the muons in the two scenarios are quite different due to the magnetized wall, which deflects the muons and spreads them over the whole detector area.
Because of this, the number of hits in the different subdetectors are proportional to the effective detector areas.

Regarding the timing of the muons, it was shown that the primary muons are created instantaneously after the bunch passing the material.
Despite that the hit time distributions of the primary muons and the produced shower particles made clear that the detector registers hits up to about \unit{100}{ns} after the bunch crossing.

The most important objective of this study was the study of the SiD occupancy and the number of dead cells caused by the muon background.
Although the occupancy in the vertex and tracker detectors is way below the critical value, the occupancy in the ECAL endcaps for example almost reaches this critical limit for the shielding scenario without the magnetized wall.
It was found out that the reason for this is not only the higher muon rate in this scenario, but also the spatial distribution of the muons which are concentrated in a small area.

Overall, with the shown evaluation of the muons from the current MUCARLO simulations, the SiD group would prefer to have the magnetized wall, in order to keep the occupancy from the backgrounds as low as possible.

To improve this study, the PACMAN geometry is recommended to be included in the SiD geometry, which will effect not only the simulations of the muon spoiler background but also all other machine backgrounds.
Finally, the BDS muon background will be used for further occupancy studies together with different other background sources to see the overall occupancy from all background sources.

