\section{The Simulation of the Muon Background with MUCARLO}
\label{MUCARLO}

The simulation code MUCARLO is based on Fortran, and was originally written by Gary Feldman (SLAC).
Over the years it has been expanded, and is used in several studies, from the study of muon shielding designs for radiation protection, to fixed target experiments at SLAC and muon background simulation studies for the Next Linear Collider (NLC) and the ILC.\\
For the presented study, the Technical Design Report (TDR) baseline machine parameters for the ILC-500GeV are used for simulating the beam interacting with the BDS geometry and the muon collimation system.
The muons are produced in interactions of the beam halo with material in the beam lines, in which the predominant interaction is the Bethe-Heilter process:\\
\textgamma + Z $\rightarrow$ Z' + \murm\textsuperscript{+}\murm\textsuperscript{-}\\
The muon production by direct annihilation of the positrons with atomic electrons is also taken into account.\cite[sec. 2]{Mucarlo}
For tracking the halo particles, the tool Turtle\cite{Turtle} is used.\\
The results from the MUCARLO simulations can be seen in Table~\ref{tab:MuonRates}, listing the number of muons reaching the interaction region for the two shielding scenarios and for the case of not having any muon shielding system.
The calculated muon rate is based on a halo population of 10\textsuperscript{-3}, which is more than ten times larger than expected from ring scattering calculations.
This estimation corresponds to the worst halo measured at the Stanford Linear Collider (SLC), and is therefore used as a worst-case scenario.

\begin{table}
\caption{MUCARLO results: The number of muons hitting a detector with radius of \unit{6.5}{m} in the different shielding scenarios.}
\label{tab:MuonRates}
\centering
\begin{tabularx}{\textwidth}{ll}
\hline\hline
\textbf{Scenario} & \textbf{Number of muons in a detector with 6.5m radius}\\
\hline
\cline{1-2}
\hline
 No Spoilers & 130 muons/bunch crossing\\
 5 Spoilers& 4.3 muons/bunch crossing\\
 5 Spoilers + Wall & 0.68 muons/bunch crossing\\
\hline\hline
\end{tabularx}
\end{table}