\section{The Simulation of the Muon Background with MUCARLO}
\label{MUCARLO}

The simulation code MUCARLO is based on Fortran, and was written by Lewis Keller.
For simulating the beam interacting with the BDS geometry and the muon collimation system, the TDR baseline machine parameters for the ILC500 are used.
The muons are produced predominantly by the Bethe-Heilter process:\\
\textgamma + Z $\rightarrow$ Z' + \murm\textsuperscript{+}\murm\textsuperscript{-}\\
The muon production by direct annihilation of the positrons with atomic electrons is also taken into account.\cite[sec. 2]{Mucarlo}
For tracking the halo particles, the tool Turtle\cite{Turtle} is used.\\
The results from the MUCARLO simulations can be seen in Table~\ref{tab:MuonRates}, listing the number of muons reaching the interaction region for the two shielding scenarios and for the case of not having any muon shielding system.

\begin{table}
\caption{MUCARLO results: The number of muons hitting a detector with radius of \unit{6.5}{m} in the different shielding scenarios.}
\label{tab:MuonRates}
\centering
\begin{tabularx}{\textwidth}{ll}
\hline\hline
\textbf{Scenario} & \textbf{Number of muons in a detector with 6.5m radius}\\
\hline
\cline{1-2}
\hline
 No Spoilers & 130 muons/bunch crossing\\
 5 Spoilers& 4.3 muons/bunch crossing\\
 5 Spoilers + Wall & 0.68 muons/bunch crossing\\
\hline\hline
\end{tabularx}
\end{table}