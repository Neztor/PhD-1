\section{Summary, Conclusions, and Outlook}
The pair background envelopes in the SiD detector were studied by using the up-to-date \guineapig data set of one bunch crossing for the ILC500GeV parameters, and taking the pair particle 4-vector as input to a Helix algorithm that calculates the helix tracks under certain assumptions.
The presented results from this study suggest that the ILC community could think about reducing the beam pipe radius by at least \unit{2}{mm} without exposing material to the pair background.
The SiD group would then have the opportunity to either add another vertex detector layer with a smaller radius or to shrink the radius of the existing vertex detector layers.
Whilst this opens the possibility for improvements in physics event reconstruction through c-tagging for example, further studies of the level of background and synchrotron radiation at such
radii need to be performed.
