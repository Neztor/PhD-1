\section{Summary and conclusion}
By looking at the luminosity spectra generated for this study, the expected rise in luminosity for the proposed ILC250 parameter sets could be confirmed.
The total luminosity resulting from the new official beam parameter set (A) is increased by about \SI{97}{\percent} with respect to the original beam parameters, leading to a new value of \SI{1.62e34}{\per\centi\meter\squared\per\second}.
The background studies indicate that the new beam parameters increase the \Pep\Pem pair background density and the SiD vertex detector occupancy by only a factor of about 2-3 in comparison to the original TDR beam parameters.
As for a precision detector the aim is to keep the overall ratio of so-called ``dead'' cells with a full buffer below a per mill level, every rise in the occupancy level can be critical.
However, although the density of the pair background close to the interaction point rises, the normalized vertex detector occupancy for the new set is well below 10$^{-4}$.
The highest occupancy is reached in the inner most layer with a fraction of dead cells of about \num{5e-5}, assuming a buffer depth of four.\\
The SiD Optimization group is confident that this can be accommodated in the design of the SiD vertex detector without a loss in precision for the physics studies.
Therefore, the SiD group welcomes the decision to change the beam parameters in order to gain higher luminosities and to strengthen the physics program.