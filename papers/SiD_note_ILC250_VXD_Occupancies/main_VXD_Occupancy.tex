\pdfoutput=1 % for arXiv to use pdflatex
\documentclass[12pt]{article}

%% The graphicx package provides the includegraphics command.
\usepackage{graphicx}
%% The amssymb package provides various useful mathematical symbols
\usepackage{amsmath}
\usepackage{amssymb}
%% The lineno packages adds line numbers. Start line numbering with
%% \begin{linenumbers}, end it with \end{linenumbers}. Or switch it on
%% for the whole article with \linenumbers after \end{frontmatter}.
\usepackage{lineno}
\usepackage{url}
\usepackage{xspace,multicol}
\usepackage{siunitx}
\usepackage{subcaption}
\usepackage{color, colortbl}
\usepackage{units}
\usepackage{ragged2e}
\usepackage{array}
\usepackage{tabularx}
\usepackage{authblk}
\usepackage{heppennames}
\usepackage{feynmp}
\usepackage{libertine}
% \usepackage[libertine]{newtxmath}
\usepackage{textgreek}
\DeclareGraphicsRule{*}{mps}{*}{}

\newcommand{\guineapig}{GuineaPig\xspace}
\newcommand{\sid}{SiD\xspace}
\newcommand{\electron}{e\textsuperscript{-}\xspace}
\newcommand{\positron}{e\textsuperscript{+}\xspace}
\newcommand{\micron}{\ensuremath{\murm\mathrm{m}\xspace}}

\newcommand{\murm}{%
  \ifmmode
    \mathchoice
        {\hbox{\normalsize\textmu}}
        {\hbox{\normalsize\textmu}}
        {\hbox{\scriptsize\textmu}}
        {\hbox{\tiny\textmu}}%
  \else
    \textmu
  \fi
}


\usepackage{fixltx2e}


\begin{document}

%% Title, authors and addresses

\title{Impact of the new ILC250 Parameter Sets\\on the \sid Vertex Detector Occupancy arrising from \positron\electron Pair Background\vspace*{0.3cm}\\{\normalsize \sid Note}}

\author[1,2]{Anne Sch\"utz}

\affil[1]{\normalsize Karlsruhe Institute of Technology (KIT), Department of Physics, Institute of Experimental Nuclear Physics (IEKP), Wolfgang-Gaede-Str. 1, 76131 Karlsruhe}
\affil[2]{\normalsize Deutsches Elektronen-Synchrotron (DESY), Notkestr. 85, 22607 Hamburg}

\maketitle

%%
%% Start line numbering here if you want
%%
%\linenumbers

\begin{abstract}
%% Text of abstract
A new change request has been made for changing the beam parameters of the ILC250 stage in order to increase the luminosity.
Due to the reduced horizontal emittance of the new parameter sets, a larger electron positron pair background is to be expected which is arrising from increased beam-beam interactions.

This proceeding presents for the first time a study of these processes in a detailed simulation, which shows that these pair background particles appear at angles that extend to the inner layers of the detector.
The full data set of pairs produced in one bunch crossing was used to calculate the helix tracks, which the particles form in the solenoid field of the SiD detector.
The results suggest to further study the reduction of the beam pipe radius and therefore to either add another SiD vertex detector layer, or reduce the radius of the existing vertex detector layers, without increasing the detector occupancy significantly.
This has to go along with additional studies whether the improvement in physics reconstruction methods, like c-tagging, is worth the increased background level at smaller radii.
\end{abstract}


%% main text
\section{Introduction}
\label{sec:introduction}
Since the pair background scatters on material and irradiates it over time, the ideal distance between the area of high density pair background and the beam pipe can be decided by studying the envelopes of the pair background helixes in the magnetic solenoid field.
Takashi Maruyama has previously done this study with different beam parameters~\cite{Takashi_plot}, but with applying cuts to the data set because of limited CPU power~\cite{Takashi}.
The plots shown in this proceeding, in contrast, were done with a full pair background data set from the most recent simulations without any cuts applied.\\
With the beam pipe radius of \unit{1.2}\,{cm} in the immediate interaction point region and the beam pipe increasing in radius in a cone shape, the fraction of pairs leaving the beam pipe can be calculated in order to convey an understanding of how many pair background particles enter the SiD detector and interact with the material. 

%\input{VXD_SiD}
%\section{Summary and conclusion}
By looking at the luminosity spectra generated for this study, the expected increase in luminosity for the new ILC250 parameter sets could be confirmed.
The background studies indicate that the new official beam parameter set (set (A)) increases the \Pep\Pem pair background density and the SiD vertex detector occupancy by only a factor of about 2-3 in comparison to the original TDR beam parameters.
Although the density of the pair background close to the interaction point rises, the normalized vertex detector occupancy for the new set is well below 10$^{-4}$.
The SiD Optimization group is confident that this can be accommodated in the design of the SiD vertex detector without a loss in precision for the physics studies.


\section*{Acknowledgments}
The author would like to thank Takashi Murayama (SLAC) for useful discussions and clarifications of his work in \cite{Takashi_plot}.

%\section{Appendix}


%% References with bibTeX database:

\bibliographystyle{unsrt}
%\bibliographystyle{model1-num-names} %Doesn't work well if you also cite webpages (Confluence page)
\bibliography{bibliography.bib}

\end{document}
