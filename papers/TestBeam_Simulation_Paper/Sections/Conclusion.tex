\section{Conclusion}
 The test beam simulation toolkit and the possible studies of the beam attributes and their dependency on the beam line geometry will serve as a key input for future test beam line improvements.
 
 As one of very few test beam facilities in the world, the DESY-II test beam facility provides users from all over the world with three high rate particle beams in the GeV range. For gaining detailed understanding of the test beam generation and its dependency on the beam attributes, a \geant Monte Carlo simulation of the test beam lines was set up.\\
A realistic modelling of the beam bunch of the DESY-II synchrotron as well as the very precise geometry description of all test beam line components were needed to implement this full simulation tool. The magnetic field strength of the simulated test beam dipole magnet was varied to examine the dependency on the final beam energy and its spread.\\
The analysis of the simulated data also yielded knowledge about the particle fluxes of primary and secondary particles, so that the  simulation of the test beam lines already provided essential input for the 2014 shutdown maintenance work. Additional shieldings were built up for stopping undesired particles from continuing on the test beam lines and affecting the energy distribution and beam purity of the final test beams.\\
The set up simulation will be used as a fully flexible \geant Monte Carlo tool for further simulation studies of the DESY-II test beam lines. Due to its ease of use, more detailed studies are now possible to find possible optimisations of the geometry of the test beam lines. The information gained will then be useful for the shutdown activities and DESY-II test beam upgrades in following years. 