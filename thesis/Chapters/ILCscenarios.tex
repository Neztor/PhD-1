\chapter{Impact of the Backgrounds on the Detectors in different ILC running scenarios}
\label{EffectDetectors}
The previous chapters introduced and discussed studies of several different background sources that occur at a linear collider.
The impact that these background sources have on the detectors will be discussed in detail for different scenarios.
First, the results for nominal ILC beam and machine parameters will be presented, followed by the results for different beam orbit specifications.
In the end, it will be shown that also different machine parameters have different effects on the outcome.

\section{Results for nominal beam and machine parameters}
\label{EffectDetectors:Nominal}
\subsection{Hit maps of the SiD subdetectors}
\label{EffectDetectors:hitmaps}
\subsection{Occupancy studies and buffer depth}
\label{EffectDetectors:occupancy}
\subsection{Hit time distributions}
\label{EffectDetectors:hittime}
\subsection{Resulting Dosis of the detectors}
\label{EffectDetectors:dosis}

\section{Results for different beam orbit specifications}
\label{EffectDetectors:BeamOrbit}

\subsection{Beam orbit deviating from the ideal nominal specifications}
\label{EffectDetectors:BeamOrbit:otherspecs}
\subsection{Resulting Dosis of the detectors}
\label{EffectDetectors:BeamOrbit:dosis}

\section{Results for different machine parameters}
\label{EffectDetectors:MachineParameters}
\subsection{Continuous wave mode}
\label{EffectDetectors:MachineParameters:CW}
In the Continuous Wave (CW) mode, the particle beam is accelerated by an electromagnetic continuous wave with constant amplitude and frequency.
\subsection{Beam pipe radius}
\label{EffectDetectors:MachineParameters:beampipe}
\subsection{Bunch spacings}
\label{EffectDetectors:MachineParameters:bunchspacing}
\subsection{TeV-Upgrade}
\label{EffectDetectors:MachineParameters:upgrade}
\subsection{Resulting Dosis of the detectors}
\label{EffectDetectors:MachineParameters:dosis}

