\chapter{Accelerator Physics of Linear Colliders}
\label{LinearColliderPhysics}

\begin{chapterabstract}
Since the invention of the first particle accelerator in 1930, many new forms of accelerators were discovered and developed over the years. 
Even though they may differ in shape and form, they all rely on the same principles. 
After a brief introduction of these principles of accelerator physics, there will be a description of the two classes of particle accelerators which are mainly used for high-energy physics nowadays: linear and circular colliders. 
By looking at their advantages and disadvantages, the differences between them will be elaborated.
\end{chapterabstract}
\newline

In 1930, J. D. Cockcroft and E. T. S. Walton constructed the first particle accelerator, in order to probe the nuclei of lithium atoms with protons accelerated to several hundred keV. 
They found that the key in particle acceleration lies in electrostatics.

\section{Principles of particle acceleration}
\label{AcceleratorPhysics}
Charged particles are accelerated inside an electric field. 
In a time independent electric field $\vec{E}$ with potential $U$, the particle with charge $q$ experiences a change in its kinetic energy by passing through this electric field.
\begin{equation}
 \Delta E = q \int \vec{E}d\vec{r} = qU
\end{equation}
The particle's charge expressed in the elementary charge e is simply multiplied by the electric potential in Volts to calculate the gain or loss of the particle's kinetic energy. 
Out of convenience, the unit for energy in particle physics is therefore eV.\\
By this logic a particle is accelerated to higher energies by applying higher and higher electrostatic fields. 
Exactly this was done in the beginning of particle accelerators by increasing the voltage applied to a capacitor and shooting the charged particle through. 
Unfortunately, there are limits to the amount of voltage that can be applied before an electric breakdown.
The solution seems simple: several capacitors in a row.
But again, this can not be done with electrostatic capacitors since the field gradient between two different capacitors is directed in the opposite direction and would decelerate the particle again.
A solution is quickly found by using time dependent electric fields.\\
In simple terms, the key principles of a linear collider is thereby already explained.
Better accelerating structures, such as drift chambers and later on superconducting radio frequency (RF) cavities, were developed over the years.
The first linear accelerator using normal conducting drift chambers of varying lengths was invented by Rolf Wider\o e at the university of Karlsruhe in Germany.
Particle accelerators over the world use nowadays mainly the superconducting RF cavities since their electric resistance is minimal due to their superconducting nature.
%TODO: Resistance or Resisitivity?
%TODO: picture of RF cavities

Rolf Wider\o e did not only build the first linear accelerator, he much more invented the betatron, as the first accelerating structure using electromagnetic fields.

\section{Linear colliders in comparison to circular colliders}
\label{Linear-Circular}
