%%%%%%%%%%%%%%%%%%%%%%%%%%%%%%%%%%%%%%%%%%%%%%%%%%
%%%%%%%%%%%%%%%%%%%%%%%%%%%%%%%%%%%%%%%%%%%%%%%%%%
%%%
%%% here all macros and definitions concerning layout will be gathered
%%%
%%%%%%%%%%%%%%%%%%%%%%%%%%%%%%%%%%%%%%%%%%%%%%%%%%
%%%%%%%%%%%%%%%%%%%%%%%%%%%%%%%%%%%%%%%%%%%%%%%%%%

%%%%%%%%%%%%%%%%%%%%%%%%%%%%%%%%%%%%%%%%%%%%%%%%%%
%%% description label font
%%%%%%%%%%%%%%%%%%%%%%%%%%%%%%%%%%%%%%%%%%%%%%%%%%
%%% When using helvetic as font, the description labels become to
%%% large, so let's use the standard text font and make it bold (which
%%% i like better anyways...)
\setkomafont{descriptionlabel}{\normalfont\bfseries}

%%%%%%%%%%%%%%%%%%%%%%%%%%%%%%%%%%%%%%%%%%%%%%%%%%
%%% hyperlinks and bookmarks menu for online reading
%%%%%%%%%%%%%%%%%%%%%%%%%%%%%%%%%%%%%%%%%%%%%%%%%%
\hypersetup{bookmarksopen=true,
bookmarksnumbered=true}

%%% general unit is 1 cm
%\setlength{\unitlength}{1cm}

%%%%%%%%%%%%%%%%%%%%%%%%%%%%%%%%%%%%%%%%%%%%%%%%%%
%%% the abstract before each chapter telling the reader
%%% what will be described here
%%%%%%%%%%%%%%%%%%%%%%%%%%%%%%%%%%%%%%%%%%%%%%%%%%
\newenvironment{chapterabstract}{\itshape}{}

%%%%%%%%%%%%%%%%%%%%%%%%%%%%%%%%%%%%%%%%%%%%%%%%%%
%%% the source file of an image
%%%%%%%%%%%%%%%%%%%%%%%%%%%%%%%%%%%%%%%%%%%%%%%%%%
\newcommand{\code}[1]{\texttt{\small{#1}}}
\newcommand{\filename}[1]{\texttt{#1}}
\newcommand{\sourcename}[1]{{\footnotesize Source:\\\filename{#1}}}

\newcommand{\packagename}[1]{{\sffamily #1}}
%%%%%%%%%%%%%%%%%%%%%%%%%%%%%%%%%%%%%%%%%%%%%%%%%%
%%% shortcuts to refer to environments and sections
%%%%%%%%%%%%%%%%%%%%%%%%%%%%%%%%%%%%%%%%%%%%%%%%%%
%\newcommand{\citep}[2][]{\mbox{Refs.~\cite[#2]{#1}}}
\newcommand{\pref}[1]{\mbox{page \pageref{#1}}}
\newcommand{\Eqref}[1]{\mbox{Eq.~(\ref{#1})}}
\newcommand{\eqrefp}[1]{\mbox{Eqs.~(\ref{#1})}}
\newcommand{\tabref}[1]{\mbox{Tab.~\ref{#1}}}
\newcommand{\tabrefp}[1]{\mbox{Tabs.~\ref{#1}}}
\newcommand{\figref}[1]{\mbox{Fig.~\ref{#1}}}
\newcommand{\figrefp}[1]{\mbox{Figs.~\ref{#1}}}
\newcommand{\secref}[1]{\mbox{Sec.~\ref{#1}}}
\newcommand{\secrefp}[1]{\mbox{Secs.~\ref{#1}}}
\newcommand{\chapref}[1]{\mbox{Chap.~\ref{#1}}}
\newcommand{\chaprefp}[1]{\mbox{Chaps.~\ref{#1}}}

%%%%%%%%%%%%%%%%%%%%%%%%%%%%%%%%%%%%%%%%%%%%%%%%%%
%%% layout for caption and captionlabels
%%%%%%%%%%%%%%%%%%%%%%%%%%%%%%%%%%%%%%%%%%%%%%%%%%
\renewcommand{\captionfont}{\itshape}
\renewcommand{\captionlabelfont}{\bfseries \upshape}
\setlength{\captionmargin}{10pt}

%%%%%%%%%%%%%%%%%%%%%%%%%%%%%%%%%%%%%%%%%%%%%%%%%%
%%% create boxed formulas within a math environment
%%% carriage return is allowed in the source
%%% (this is not the case when using the AMS command \boxed)
%%%%%%%%%%%%%%%%%%%%%%%%%%%%%%%%%%%%%%%%%%%%%%%%%%
\newcommand{\mathbox}[1]{
    \fbox{ $ \displaystyle #1 $ }
}


%%%%%%%%%%%%%%%%%%%%%%%%%%%%%%%%%%%%%%%%%%%%%%%%%%
%%% a single named theorem environment for the
%%% bloch-floquet theorem
%%%%%%%%%%%%%%%%%%%%%%%%%%%%%%%%%%%%%%%%%%%%%%%%%%
\newtheorem*{BlochFloquetTheorem}{Bloch-Floquet Theorem}
\newtheorem{theorem}{Theorem}


%%%%%%%%%%%%%%%%%%%%%%%%%%%%%%%%%%%%%%%%%%%%%%%%%%
%%% set the style of the bibliography
%%%%%%%%%%%%%%%%%%%%%%%%%%%%%%%%%%%%%%%%%%%%%%%%%%
\bibliographystyle{unsrtnat}

%%%%%%%%%%%%%%%%%%%%%%%%%%%%%%%%%%%%%%%%%%%%%%%%%%
%%% a general purpose width, set where needed and
%%% helps typesetting mainly math equations where
%%% equal box sizes are needed
%%%%%%%%%%%%%%%%%%%%%%%%%%%%%%%%%%%%%%%%%%%%%%%%%%
\newlength{\auxwidth} 

%%%%%%%%%%%%%%%%%%%%%%%%%%%%%%%%%%%%%%%%%%%%%%%%%%
%%% width for mode pictures,
%%% which will be set side by side. this width is
%%% set when needed
%%%%%%%%%%%%%%%%%%%%%%%%%%%%%%%%%%%%%%%%%%%%%%%%%%
\newlength{\modewidth} 

%%%%%%%%%%%%%%%%%%%%%%%%%%%%%%%%%%%%%%%%%%%%%%%%%%
%%% separator line widths
%%%%%%%%%%%%%%%%%%%%%%%%%%%%%%%%%%%%%%%%%%%%%%%%%%
\setheadtopline{2pt}
\setheadsepline{0.5pt}
\setfootsepline{0.5pt}
\setfootbotline{2pt}

\delimitershortfall=-2pt

%%%%%%%%%%%%%%%%%%%%%%%%%%%%%%%%%%%%%%%%%%%%%%%%%%
%%% width for one, two, three, four plots per line
%%%%%%%%%%%%%%%%%%%%%%%%%%%%%%%%%%%%%%%%%%%%%%%%%%
\newlength{\plotwidthOne} 
\setlength{\plotwidthOne}{0.975\textwidth} 
\newlength{\plotwidthTwo} 
\setlength{\plotwidthTwo}{0.477\textwidth} 
\newlength{\plotwidthThree} 
\setlength{\plotwidthThree}{0.31\textwidth} 
\newlength{\plotwidthThreeTwo} 
\setlength{\plotwidthThreeTwo}{0.642\textwidth} 
\newlength{\plotwidthThreeOne} 
\setlength{\plotwidthThreeOne}{0.31\textwidth} 
\newlength{\plotwidthFour} 
\setlength{\plotwidthFour}{0.227\textwidth} 

\newcommand{\includesingleplot}[1]{
  \parbox{\singleplotwidth}{
    \includegraphics[width=\plotwidthTwo]{#1}\\
    \sourcename{#1.agr}
  }
}
\newcommand{\includedoubleplot}[1]{
  \parbox{\doubleplotwidth}{
    \includegraphics[width=\plotwidthOne]{#1}\\
    \sourcename{#1.agr}
  }
}
\newcommand{\includeannotatedplot}[3]{
  \parbox{#1}{
    \centering    
    \includegraphics[width=#1]{#2}\\
    #3
  }
}

%%% define a new floating environment used for tables and figures side by side
\newfloat{hybridfloat}{ht}{hybridfloat}


%%%%%%%%%%%%%%%%%%%%%%%%%%%%%%%%%%%%%%%%%%%%%%%%%%%%%%%%%%%%
%%%%%%%%%%%%%%%%%%%%%%%%%%%%%%%%%%%%%%%%%%%%%%%%%%%%%%%%%%%%
%%%%%%%                            %%%%%%%%%%%%%%%%%%%%%%%%%
%%%%%   color definitions            %%%%%%%%%%%%%%%%%%%%%%%
%%%%%%                             %%%%%%%%%%%%%%%%%%%%%%%%%
%%%%%%%%%%%%%%%%%%%%%%%%%%%%%%%%%%%%%%%%%%%%%%%%%%%%%%%%%%%%
%%%%%%%%%%%%%%%%%%%%%%%%%%%%%%%%%%%%%%%%%%%%%%%%%%%%%%%%%%%%
\definecolor{greenyellow}   {cmyk}{0.15, 0   , 0.69, 0   }
\definecolor{yellow}        {cmyk}{0   , 0   , 1   , 0   }
\definecolor{goldenrod}     {cmyk}{0   , 0.10, 0.84, 0   }
\definecolor{dandelion}     {cmyk}{0   , 0.29, 0.84, 0   }
\definecolor{apricot}       {cmyk}{0   , 0.32, 0.52, 0   }
\definecolor{peach}         {cmyk}{0   , 0.50, 0.70, 0   }
\definecolor{melon}         {cmyk}{0   , 0.46, 0.50, 0   }
\definecolor{yelloworange}  {cmyk}{0   , 0.42, 1   , 0   }
\definecolor{orange}        {cmyk}{0   , 0.61, 0.87, 0   }
\definecolor{burntorange}   {cmyk}{0   , 0.51, 1   , 0   }
\definecolor{bittersweet}   {cmyk}{0   , 0.75, 1   , 0.24}
\definecolor{redorange}     {cmyk}{0   , 0.77, 0.87, 0   }
\definecolor{mahogany}      {cmyk}{0   , 0.85, 0.87, 0.35}
\definecolor{maroon}        {cmyk}{0   , 0.87, 0.68, 0.32}
\definecolor{brickred}      {cmyk}{0   , 0.89, 0.94, 0.28}
\definecolor{red}           {cmyk}{0   , 1   , 1   , 0   }
\definecolor{orangered}     {cmyk}{0   , 1   , 0.50, 0   }
\definecolor{rubinered}     {cmyk}{0   , 1   , 0.13, 0   }
\definecolor{wildstrawberry}{cmyk}{0   , 0.96, 0.39, 0   }
\definecolor{salmon}        {cmyk}{0   , 0.53, 0.38, 0   }
\definecolor{carnationpink} {cmyk}{0   , 0.63, 0   , 0   }
\definecolor{magenta}       {cmyk}{0   , 1   , 0   , 0   }
\definecolor{violetred}     {cmyk}{0   , 0.81, 0   , 0   }
\definecolor{rhodamine}     {cmyk}{0   , 0.82, 0   , 0   }
\definecolor{mulberry}      {cmyk}{0.34, 0.90, 0   , 0.02}
\definecolor{redviolet}     {cmyk}{0.07, 0.90, 0   , 0.34}
\definecolor{fuchsia}       {cmyk}{0.47, 0.91, 0   , 0.08}
\definecolor{lavender}      {cmyk}{0   , 0.48, 0   , 0   }
\definecolor{thistle}       {cmyk}{0.12, 0.59, 0   , 0   }
\definecolor{orchid}        {cmyk}{0.32, 0.64, 0   , 0   }
\definecolor{darkorchid}    {cmyk}{0.40, 0.80, 0.20, 0   }
\definecolor{purple}        {cmyk}{0.45, 0.86, 0   , 0   }
\definecolor{plum}          {cmyk}{0.50, 1   , 0   , 0   }
\definecolor{violet}        {cmyk}{0.79, 0.88, 0   , 0   }
\definecolor{royalpurple}   {cmyk}{0.75, 0.90, 0   , 0   }
\definecolor{blueviolet}    {cmyk}{0.86, 0.91, 0   , 0.04}
\definecolor{periwinkle}    {cmyk}{0.57, 0.55, 0   , 0   }
\xdefinecolor{cadetblue}     {cmyk}{0.62, 0.57, 0.23, 0   }
\xdefinecolor{cornflowerblue}{cmyk}{0.65, 0.13, 0   , 0   }
\xdefinecolor{midnightblue}  {cmyk}{0.98, 0.13, 0   , 0.43}
\xdefinecolor{navyblue}      {cmyk}{0.94, 0.54, 0   , 0   }
\xdefinecolor{royalblue}     {cmyk}{1   , 0.50, 0   , 0   }
\definecolor{blue}          {cmyk}{1   , 1   , 0   , 0   }
\xdefinecolor{darkblue}     {cmyk}{1   , 1   , 0   , 0.5 }
\definecolor{cerulean}      {cmyk}{0.94, 0.11, 0   , 0   }
\definecolor{cyan}          {cmyk}{1   , 0   , 0   , 0   }
\definecolor{processblue}   {cmyk}{0.96, 0   , 0   , 0   }
\definecolor{skyblue}       {cmyk}{0.62, 0   , 0.12, 0   }
\definecolor{turquoise}     {cmyk}{0.85, 0   , 0.20, 0   }
\definecolor{tealblue}      {cmyk}{0.86, 0   , 0.34, 0.02}
\definecolor{aquamarine}    {cmyk}{0.82, 0   , 0.30, 0   }
\definecolor{bluegreen}     {cmyk}{0.85, 0   , 0.33, 0   }
\definecolor{emerald}       {cmyk}{1   , 0   , 0.50, 0   }
\definecolor{junglegreen}   {cmyk}{0.99, 0   , 0.52, 0   }
\definecolor{seagreen}      {cmyk}{0.69, 0   , 0.50, 0   }
\definecolor{green}         {cmyk}{1   , 0   , 1   , 0   }
\definecolor{forestgreen}   {cmyk}{0.91, 0   , 0.88, 0.12}
\definecolor{pinegreen}     {cmyk}{0.92, 0   , 0.59, 0.25}
\definecolor{limegreen}     {cmyk}{0.50, 0   , 1   , 0   }
\definecolor{yellowgreen}   {cmyk}{0.44, 0   , 0.74, 0   }
\definecolor{springgreen}   {cmyk}{0.26, 0   , 0.76, 0   }
\definecolor{olivegreen}    {cmyk}{0.64, 0   , 0.95, 0.40}
\definecolor{rawsienna}     {cmyk}{0   , 0.72, 1   , 0.45}
\definecolor{sepia}         {cmyk}{0   , 0.83, 1   , 0.70}
\definecolor{brown}         {cmyk}{0   , 0.81, 1   , 0.60}
\definecolor{tan}           {cmyk}{0.14, 0.42, 0.56, 0   }
\definecolor{gray}          {cmyk}{0   , 0   , 0   , 0.50}
\definecolor{black}         {cmyk}{0   , 0   , 0   , 1   }
\definecolor{white}         {cmyk}{0   , 0   , 0   , 0   } 
\definecolor{darkgray}      {cmyk}{0   , 0   , 0   , 0.75}


%%%%%%%%%%%%%%%%%%%%%%%%%%%%%%%%%%%%%%%%%%%%%%%%%%%%%%%%%%%%
%%%%%%%%%%%%%%%%%%%%%%%%%%%%%%%%%%%%%%%%%%%%%%%%%%%%%%%%%%%%
%%%%%%%                            %%%%%%%%%%%%%%%%%%%%%%%%%
%%%%%   chapter headings             %%%%%%%%%%%%%%%%%%%%%%%
%%%%%%                             %%%%%%%%%%%%%%%%%%%%%%%%%
%%%%%%%%%%%%%%%%%%%%%%%%%%%%%%%%%%%%%%%%%%%%%%%%%%%%%%%%%%%%
%%%%%%%%%%%%%%%%%%%%%%%%%%%%%%%%%%%%%%%%%%%%%%%%%%%%%%%%%%%%

%%% INFO: when using different fonts, the chapter headings may look
%%% different and the given sizes do no longer apply!

\colorlet{chapter}{darkblue}
\addtokomafont{chapter}{\color{chapter}}

\makeatletter% siehe De-TeX-FAQ
\renewcommand*{\chapterformat}{%
\begingroup% damit \unitlength-Änderung lokal bleibt
\setlength{\unitlength}{1mm}%
\begin{picture}(20,40)(0,5)%
\setlength{\fboxsep}{0pt}%
%\put(0,0){\framebox(20,30){}}%
%\put(0,20){\makebox(20,20){\rule{20\unitlength}{20\unitlength}}}%
\put(20,15){\line(1,0){\dimexpr
\textwidth-20\unitlength\relax\@gobble}}%
\put(0,0){\makebox(20,20)[r]{%
\fontsize{28\unitlength}{1}\selectfont\thechapter
%\kern-.04em% Ziffer in der Zeichenzelle nach rechts schieben
}}%
\put(20,15){\makebox(\dimexpr
\textwidth-20\unitlength\relax\@gobble,\ht\strutbox\@gobble)[l]{%
\ \normalsize\color{black}\chapapp~\thechapter\autodot
}}%
\end{picture} % <- Leerzeichen ist hier beabsichtigt!
\endgroup
}%
\makeatother

%%% end of chapter headings
%%%%%%%%%%%%%%%%%%%%%%%%%%%%%%%%%%%%%%%%%%%%%%%%%%%%%%%%%%%%
%%%%%%%%%%%%%%%%%%%%%%%%%%%%%%%%%%%%%%%%%%%%%%%%%%%%%%%%%%%%



%%%%%%%%%%%%%%%%%%%%%%%%%%%%%%%%%%%%%%%%%%%%%%%%%%%%%%%%%%%%
%%%%%%%%%%%%%%%%%%%%%%%%%%%%%%%%%%%%%%%%%%%%%%%%%%%%%%%%%%%%
%%%%%%%                            %%%%%%%%%%%%%%%%%%%%%%%%%
%%%%%   boxed equations             %%%%%%%%%%%%%%%%%%%%%%%
%%%%%%                             %%%%%%%%%%%%%%%%%%%%%%%%%
%%%%%%%%%%%%%%%%%%%%%%%%%%%%%%%%%%%%%%%%%%%%%%%%%%%%%%%%%%%%
%%%%%%%%%%%%%%%%%%%%%%%%%%%%%%%%%%%%%%%%%%%%%%%%%%%%%%%%%%%%

%%%
\newcommand*\widefbox[1]{\fbox{\hspace{1em}#1\hspace{1em}}}

%%% Simple box containing subequations
\newcommand*{\boxedsubeq}[2]{%
\begin{subequations}\label{#1}%
\begin{empheq}[box=\widefbox]{align}%
#2
\end{empheq}%
\end{subequations}
}

%%% Very fancy box containing subequations
\definecolor{shadecolor}{cmyk}{0,0,0.21,0}%
\definecolor{headercolor}{cmyk}{0,0,0.5,0}%{0.25,0,0,0}
%\colorlet{shadecolor}{gray}

\newsavebox{\mysaveboxM}
\newsavebox{\mysaveboxT}

\newcommand*{\fancyBoxedSubeq}[2][Example]{%
 \sbox{\mysaveboxM}{#2}%
 \sbox{\mysaveboxT}{\fcolorbox{black}{headercolor}{#1}}%
 \sbox{\mysaveboxM}{%
  \parbox[b][\ht\mysaveboxM+.5\ht\mysaveboxT+.5\dp\mysaveboxT][b]{%
    \wd\mysaveboxM}{#2}%
 }%
\sbox{\mysaveboxM}{%
  \fcolorbox{black}{shadecolor}{%
     \makebox[\linewidth-5em]{\usebox{\mysaveboxM}}%
  }%
 }%
  \usebox{\mysaveboxM}%
  \makebox[0pt][r]{%
    \makebox[\wd\mysaveboxM][c]{%
       \raisebox{\ht\mysaveboxM-0.5\ht\mysaveboxT
                 +0.5\dp\mysaveboxT-0.5\fboxrule}{\usebox{\mysaveboxT}}%
    }%
  }%
}
