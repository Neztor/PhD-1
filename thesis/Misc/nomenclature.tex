\chapter*{Nomenclature} 
\addcontentsline{toc}{chapter}{Nomenclature} 
\paragraph{Beam Delivery System (BDS)}
The Beam Delivery System (BDS) as part of the International Linear Collider transports the beam bunches from the main linear accelerators to the interaction point, where the bunches are brought into collision.
The BDS contains the Final-Focus (FF) system, where the bunches are focused to nanometer-size.
\paragraph{E\textsubscript{CM}}
The center-of-mass energy E\textsubscript{CM} is the collision energy of a particle accelerator.
If the particles of the colliding beams have equal masses and energies E, the center-of-mass energy can be calculated to be E\textsubscript{CM} = 2E. Its unit is commonly [\si{\GeV}].
\paragraph{Final-Focus system (FF)}
The Final-Focus system (FF) is part of the Beam Delivery System (BDS) at the International Linear Collider.
It contains correction magnets that focus the beam bunches to nanometer-size and correct orbital fluctuations.
\paragraph{$\bm{\mathcal{L}}$}
The luminosity $\mathcal{L}$ of a particle accelerator is a measure of the amount of collisions that can occur. It is given as the number of collisions per area per time, with the units [\si{\per\centi\meter\squared\per\second}].
\paragraph{N}
Every beam bunch contains a certain number of particles, N. This quantity is often called ``bunch population''.
\paragraph{n\textsubscript{b}}
The number of bunches per train or pulse of an accelerator is given as n\textsubscript{b}.
\paragraph{Occupancy}
The detector occupancy is the fraction of detector pixels that are hit by particles with respect to the total number of pixels.
\paragraph{q\textsubscript{b}}
From the number of particles in a beam bunch and their charge, the bunch charge can be calculated in units of Coulomb.
\paragraph{$\bm\sigma$}
$\sigma_{x,y,z}$ gives the size of a beam bunch in the horizontal, vertical and longitudinal direction. When the beam size is given specifically at the interaction point of the accelerator, $\sigma^*$ is used.
\paragraph{$\Delta$t\textsubscript{b}}
The time between each bunch crossing in a particle accelerator.
\paragraph{X\textsubscript{0}}
The radiation length X\textsubscript{0} is a material specific constant, and is defined in two ways depending on the particle type:
\begin{itemize}
 \item The mean length after which the energy of the electron (or any other charged particle) is reduced to 1/e of the initial energy due to bremsstrahlung.
 \item 7/9 of the mean free path of photons before they pair produce.
\end{itemize}
X\textsubscript{0} is measured in [\si{\gram \per\centi\meter\squared}].
